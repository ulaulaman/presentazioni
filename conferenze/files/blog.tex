\section{Il mondo dei blog}

\begin{frame}[statistiche]
 \frametitle{Statistiche\footnote{\href{http://blog.nielsen.com/nielsenwire/online_mobile/buzz-in-the-blogosphere-millions-more-bloggers-and-blog-readers/}{nielsen}}}
 \begin{center}
  \includegraphics[width=6cm]{files/nielsen.png}
 \end{center}
\end{frame}

\subsection{Definizione}

\begin{frame}[definizione]
 \frametitle{Definizione\footnote{Walker (2006); Zivkovic (2012)}}
 \scriptsize
 \begin{block}{Jill Walker}
  Suddivide i blog in due tipologie: \emph{public intellectuals} e \emph{research blogs}
 \end{block}
\onslide<2->{
 \begin{block}{Bora Zivkovic, blog editor di \emph{Scientific American}}
  Usually it is meant to be a blog that satisfies one or more of these criteria: blog written by a scientist, blog written by a professional science writer/journalist, blog that predominantly covers science topics, blog used in a science classroom as a teaching tool, blog used for more-or-less official news and press releases by scientific societies, institutes, centers, universities, publishers, companies and other organizations.
 \end{block}}
\end{frame}

\subsection{Aggregatori e network}

\begin{frame}[aggregatori]
 \frametitle{Gli aggregatori}
  \only<1>{
   \begin{center}
    \includegraphics[width=8cm]{files/sb.jpg}
   \end{center}}
  \only<2>{
   \begin{center}
    \includegraphics[width=6cm]{files/mb.jpg}
   \end{center}}
 \only<3>{
   \begin{center}
    \includegraphics[width=8cm]{files/rb.jpg}
   \end{center}}
  \only<4>{
   \begin{center}
    \includegraphics[width=8cm]{files/scienceseeker.jpg}
   \end{center}}
  \begin{itemize}
   \item<1-> Scienceblogging
   \item<2-> Mathblogging
   \item<3-> Research Blogging
   \item<4-> ScienceSeeker
  \end{itemize}
\end{frame}

\begin{frame}[network]
 \frametitle{I network}
 \begin{center}
  \includegraphics[width=6cm]{files/sciblogs.jpg}
 \end{center}
 \begin{center}
  \includegraphics[width=6cm]{files/fs.jpg}
 \end{center}
 \begin{center}
  \includegraphics[width=6cm]{files/sciam.jpg}
 \end{center}
\end{frame}
