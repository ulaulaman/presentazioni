\section{I blog come oggetto d'indagine}
\subsection{La struttura}

\begin{frame}
\scriptsize
\frametitle{Caratteristiche\footnote{Ashlin, Ladle (2006), Kjellberg (2010), Amsen (2008)}}
 \pause\begin{block}{}
  \begin{center}\alert{permette una condivisione della conoscenza, che pu� essere sviluppata anche in modo collaborativo (blog multiutente, network, wiki)}\end{center}
 \end{block}
 \pause\begin{block}{}
  \begin{center}\alert{un modo eccellente per comunicare la passione per il proprio campo di ricerca}\end{center}
 \end{block}
 \pause\begin{block}{}
  \begin{center}\alert{permette di partecipare in qualche modo al dibattito pubblico e mantenere la cittadinanza informata}\end{center}
 \end{block}
 \pause\begin{block}{}
  \begin{center}\alert{permette di restare in contatto con colleghi lontani o di stringere nuovi contatti}\end{center}
 \end{block}
 \pause\begin{block}{}
  \begin{center}\alert{di migliorare la propria scrittura anche in funzione della stesura di un articolo di ricerca}\end{center}
 \end{block}
 \pause\begin{block}{}
  \begin{center}\alert{permette la discussione informale riguardo vari argomenti con i propri lettori che possono utilizzare i commenti}\end{center}
 \end{block}
\end{frame}

\begin{frame}[caduta]
\frametitle{La caduta delle gerarchie\footnote{Walker (2006); Bell (2012)}}
 \scriptsize
 \begin{block}{(Stephen and Harrison 768-69)}
 In a well-known case study, Zuboff (1988) documented the tension created within a corporation when a computer-based electronic communication system was installed. The openness, inclusiveness, and anonymity of computer-mediated communication was antithetical to the organization�s hierarchical structure; it facilitated the rise of democratic dialogue among workers, thereby placing stress upon traditional hierarchical roles.
 \end{block}
 \onslide<2->{
 \begin{block}{Rapporto con giornalisti e istituzioni}
  Secondo Alice Bell (2012), il rapporto con i giornalisti � un falso problema e viene trattato in maniera semplicistica.\\
 Differenti le problematiche con le istituzioni, che spesso si affidano ai comunicati stampa, non sempre redatti in maniera corretta o a volte diffusi troppo frettolosamente.
 \end{block}}
\end{frame}

\subsection{L'uso dei link}
\begin{frame}[link]
\frametitle{Creare una rete: l'uso dei link\footnote{Luz�n Marco (2009)}}
 \scriptsize
 \begin{block}{I link vengono utilizzati}
  per posizionarsi all'interno di una comunit� e costruire delle relazioni;\\
  per distribuire e organizzare le informazioni;\\
  per collaborare nella costruzione della conoscenza;\\
  per creare una identit� per il blogger e/o il suo blog;\\
  per conversare;\\
  per pubblicizzare la propria ricerca;\\
  per aumentare la visibilit� del blog.
 \end{block}
\end{frame}

\subsection{Research Blogging}
\begin{frame}[RB]
 \frametitle{Research Blogging\footnote{Groth e Gurney (2010), Shema, Bar-Ilan e Thelwall (2012)}}
 \scriptsize
  \begin{block}{La discussione scientifica tra i blog iscritti all'aggregatore}
   � pi� immediata rispetto a quella della letteratura tradizionale;\\
   � pi� contestualmente rilevante;\\
   si concentra sulla scienza di alta qualit�;\\
   si concentra sulle implicazioni non tecniche della scienza.
  \end{block}
\end{frame}

\subsubsection{La situazione italiana}
\begin{frame}
 \frametitle{Un po' di dati su RB in italiano}
 \only<1>{
 \begin{center}
  \includegraphics[width=4cm]{files/rb_italia01.jpg}
 \end{center}}
 \only<2>{
 \begin{center}
  \includegraphics[width=6cm]{files/rb_international.jpg}
 \end{center}
 dall'articolo di Shema, Bar-Ilan e Thelwall (2012)}
 \only<3>{
 \begin{center}
  \includegraphics[width=10cm]{files/rb_italia_chart.png}
 \end{center}}
 \only<4>{
 \begin{center}
  \includegraphics[width=10cm]{files/rb_italia_freq.png}
 \end{center}}
\end{frame}

\begin{frame}
 \frametitle{Una rete}
 \begin{center}
  \includegraphics[width=10cm]{files/moschettieri.jpg}
 \end{center}
\end{frame}

\begin{frame}
 \frametitle{United we stand}
 \scriptsize
 \begin{center}
  \includegraphics[width=10cm]{files/united_we_stand.jpg}
 \end{center}
 \begin{block}{da Scienceblogs.com}
  Our mission is to build a community of like-minded individuals who are passionate about science and its place in our culture, and give them a place to meet
 \end{block}
\end{frame}
