\documentclass{beamer}

\usepackage{ifpdf}
\ifpdf
\usepackage[hyperindex]{hyperref}
%\pdfadjustspacing=1
%\fi

\mode<presentation>
 {
  \usetheme{Frankfurt}
   \usecolortheme[rgb={0.36,0.54,0.66}]{structure}
   
   \definecolor{inaf}{HTML}{1D71B8}
   %\definecolor{ashgrey}{rgb}{0.7, 0.75, 0.71}
   \definecolor{autumn}{rgb}{0.7, 0.75, 0.71}
   \definecolor{autumn1}{rgb}{0.7, 0.75, 0.71}
   \definecolor{autumn2}{rgb}{0.36, 0.54, 0.66}

\setbeamercolor{alerted text}{fg=inaf!80!yellow}
\setbeamercolor*{palette primary}{fg=inaf!60!black,bg=autumn}
\setbeamercolor*{palette secondary}{fg=white!70!black,bg=autumn2}
\setbeamercolor*{palette tertiary}{bg=white!80!black,fg=autumn2}
\setbeamercolor*{palette quaternary}{fg=white,bg=autumn2}

\setbeamercolor*{sidebar}{fg=inaf,bg=autumn}

\setbeamercolor*{palette sidebar primary}{fg=inaf!10!black}
\setbeamercolor*{palette sidebar secondary}{fg=white}
\setbeamercolor*{palette sidebar tertiary}{fg=inaf!50!black}
\setbeamercolor*{palette sidebar quaternary}{fg=yellow!10!orange}

\setbeamercolor*{titlelike}{parent=palette primary}
\setbeamercolor{frametitle}{bg=autumn1}
\setbeamercolor{frametitle right}{bg=autumn}

\setbeamercolor*{separation line}{}
\setbeamercolor*{fine separation line}{}

\mode
<all>
   
   %\usecolortheme{wolverine}
   \usecolortheme{rose}
   \usefonttheme{serif}
%   \setbeamercolor{section in toc}{fg=red}
 }

\title[Edu INAF]{Edu INAF\\Il nuovo portale per la didattica dell'Istituto Nazionale di Astrofisica}
\author[G.Filippelli, L.Barbalini]{Gianluigi Filippelli, Laura Barbalini}
\date{LXI Congresso della Societ\`a Astronomica Italiana, 14/09/2017}

\usepackage[latin1]{inputenc}
\usepackage[italian]{babel}
\usepackage{times}
%
\begin{document}
%
\begin{frame}
 \titlepage
\end{frame}
%
% Cos'� e cosa fa
%
\section{Cos'� Edu INAF}
\begin{frame}[eduinaf01]
	\frametitle{Cos'� Edu INAF}	
	Il portale EDU INAF � la piattaforma per la didattica e la divulgazione dell'Istituto Nazionale di Astrofisica.\\	
	\emph{EDU INAF � stato realizzato all'interno del programma/piattaforma REAL (Risorse Educative per l'Astrofisica Laboraoriale) finanziato dal Ministero dell'Istruzione, Universit� e Ricerca (MIUR, Legge 6/2000), con la partecipazione di tutti gli Uffici di Didattica e Divulgazione dislocati nelle sedi INAF sul territorio italiano.}
\end{frame}
%
\subsection{Cosa fa Edu INAF}
\begin{frame}[eduinaf02]
	\frametitle{Cosa fa Edu INAF}
	\scriptsize
	\begin{block}{Per le scuole e gli insegnanti}
		Risorse didattiche, corsi on-line
	\end{block}
	\onslide<2->
	\begin{block}{Per le sedi INAF}
		Segnalazione attivit� ed eventi di didattica e divulgazione
	\end{block}
	\onslide<3->
	\begin{block}{Per i ricercatori}
		Divulgazione legata ai progetti scientifici
	\end{block}
\end{frame}
%
% Struttura
%
\section{Struttura}
\subsection{Homepage}
\begin{frame}[homepage]
	\frametitle{L'homepage}
	\begin{center}
		\href{http://edu.iasfbo.inaf.it/}{\includegraphics[width=10cm]{files/eduinaf_home.jpg}}
	\end{center}
	Link: \href{http://edu.inaf.it/}{\textcolor{inaf}{edu.inaf.it}}
\end{frame}
%
% Didattica
%
\subsection{Didattica}
%
\begin{frame}[didattica01]
	\frametitle{Per la didattica}
	\begin{center}
		\href{http://edu.iasfbo.inaf.it/index.php/risorse-didattiche/}{\includegraphics[width=10cm]{files/risorse.jpg}}
	\end{center}
\end{frame}
\begin{frame}[didattica02]
	\frametitle{Per la didattica}
		\begin{center}
			\href{http://edu.iasfbo.inaf.it/index.php/attivita_didattica/costruzione-del-sistema-solare-in-scala/}{\includegraphics[width=10cm]{files/sistema_solare.jpg}}
		\end{center}
\end{frame}
%
\begin{frame}[didattica03]
	\frametitle{Per la didattica}
	\begin{center}
		\href{http://edu.iasfbo.inaf.it/index.php/corsi-di-formazione/}{\includegraphics[width=10cm]{files/corsi.jpg}}
	\end{center}
\end{frame}
%
\begin{frame}[didattica04]
	\frametitle{Per la didattica}
	\begin{block}{}
		Le due sezioni forniranno agli insegnanti delle scuole italiane (e pi� in generale a tutti coloro che sono interessati alla didattica e alla divulgazione) gli strumenti per introdurre al meglio l'astronomia nei programmi scolastici, partecipare alla realizzazione di tali strumenti e avere la possibilit� di formare e migliorare le proprie competenze.
	\end{block}
\end{frame}
%
% Contenuti dinamici
%
\subsection{Eventi}
%
\begin{frame}[eventi]
	\frametitle{Gli eventi}
	\begin{center}
		\includegraphics[width=10cm]{files/eventi.jpg}
	\end{center}
	Visualizzazione per \href{http://edu.iasfbo.inaf.it/index.php/eventi/}{\textcolor{inaf}{eventi}} e \href{http://edu.iasfbo.inaf.it/index.php/eventi/luoghi/}{\textcolor{inaf}{luoghi}}
\end{frame}
%
\subsection{News}
%
\begin{frame}[news]
	\frametitle{Le news}
	\begin{center}
		\href{http://edu.iasfbo.inaf.it/index.php/category/news/}{\includegraphics[width=10cm]{files/news.jpg}}
	\end{center}
\end{frame}
%
\subsection{Rubriche}
%
\begin{frame}[rubriche01]
	\frametitle{Le rubriche}
	\begin{center}
		\href{http://edu.iasfbo.inaf.it/index.php/rubriche/}{\includegraphics[width=8cm]{files/rubriche.jpg}}
	\end{center}
\end{frame}
%
\begin{frame}[rubriche02]
	\frametitle{Le rubriche: da implementare}
	\begin{itemize}
		\item<1-> L'astronomo risponde
		\item<2-> AstroBlog
	\end{itemize}
\end{frame}
%
\subsection{Rete IRNET}
%
\begin{frame}[irnet]
	\frametitle{Rete IRNET}
	\begin{block}{}
		La rete IRNET (\emph{Italian Remote Network of Educational Telescopes}) � l'insieme dei telescopi ottici e dei radiotelescopi gestiti dall'INAF sul territorio italiano e messi a disposizione del pubblico per attivit� didattiche e divulgative (e in alcuni casi anche per ricerca scientifica).
	\end{block}
\end{frame}
%
\subsection{WebINAF}
%
\begin{frame}[webinaf]
	\frametitle{WebINAF}
	\begin{block}{}
		Pensata per essere una vera e propria sitografia, una raccolta di mini-siti didattici e divulgativi realizzati negli anni dalle sedi Inaf.
	\end{block}
\end{frame}
%
\subsection{Timeline}
%
\begin{frame}
	\frametitle{Prossimi passi}
	\begin{block}{}
		Una prima versione del sito per fine ottobre. E poi sempre maggiori contenuti con l'anno nuovo.
	\end{block}
\end{frame}
%
\section{Ringraziamenti}
%
\begin{frame}[contatti]
	\begin{block}{Grazie per averci ascoltato}
		Per informazioni scriveteci a eduinaf@inaf.it
	\end{block}
\end{frame}
\end{document}
