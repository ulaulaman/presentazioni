\documentclass{beamer}
\usepackage{chronosys}
\usepackage{tikz}

\usepackage{ifpdf}
\ifpdf
\usepackage{hyperref}
%\pdfadjustspacing=1
%\fi

\mode<presentation>
 {
  \usetheme{Frankfurt}
   \usecolortheme[rgb={0.36,0.54,0.66}]{structure}
   
   \definecolor{inaf}{HTML}{1D71B8}
   %\definecolor{ashgrey}{rgb}{0.7, 0.75, 0.71}
   \definecolor{autumn}{rgb}{0.7, 0.75, 0.71}
   \definecolor{autumn1}{rgb}{0.7, 0.75, 0.71}
   \definecolor{autumn2}{rgb}{0.36, 0.54, 0.66}
   
   \definecolor{blue}{HTML}{84CECC}
   \definecolor{gr}{HTML}{375D81}

\setbeamercolor{alerted text}{fg=inaf!80!yellow}
\setbeamercolor*{palette primary}{fg=inaf!60!black,bg=autumn}
\setbeamercolor*{palette secondary}{fg=white!70!black,bg=autumn2}
\setbeamercolor*{palette tertiary}{bg=white!80!black,fg=autumn2}
\setbeamercolor*{palette quaternary}{fg=white,bg=autumn2}

\setbeamercolor*{sidebar}{fg=inaf,bg=autumn}

\setbeamercolor*{palette sidebar primary}{fg=inaf!10!black}
\setbeamercolor*{palette sidebar secondary}{fg=white}
\setbeamercolor*{palette sidebar tertiary}{fg=inaf!50!black}
\setbeamercolor*{palette sidebar quaternary}{fg=yellow!10!orange}

\setbeamercolor*{titlelike}{parent=palette primary}
\setbeamercolor{frametitle}{bg=autumn1}
\setbeamercolor{frametitle right}{bg=autumn}

\setbeamercolor*{separation line}{}
\setbeamercolor*{fine separation line}{}

\mode
<all>
   
   %\usecolortheme{wolverine}
   \usecolortheme{rose}
   \usefonttheme{serif}
%   \setbeamercolor{section in toc}{fg=red}
 }

\title[Racconto]{Come scrivere un articolo divulgativo sulla scienza}
\author[G.Filippelli]{Gianluigi Filippelli}
\date{Liceo "C. Cavalleri", Parabiago (Milano). 08/05/2018}

\usepackage[latin1]{inputenc}
\usepackage[italian]{babel}
\usepackage{times}
%
\begin{document}
%
\begin{frame}
 \titlepage
\end{frame}
%
% Introduzione
%
\section{Raccontare la scienza}
%
\begin{frame}
	\frametitle{Consigli per raccontare la scienza}
	\begin{itemize}
		\item Scrivere di una data ricerca solo quando la si reputa interessante (non scrivere necessariamente di risultati recenti)
		\onslide<2->{\item Essere certi di aver compreso la ricerca di cui si vuole scrivere}
		\onslide<3->{\item Mostrare perch� la ricerca � interessante}
		\onslide<4->{\item Fornire dettagli su come la ricerca � stata condotta, o quanto meno dare un'idea sul metodo usato dai ricercatori}
		\onslide<5->{\item Non includere dettagli rilevanti solo per gli scienziati}
	\end{itemize}
\end{frame}
%
\begin{frame}
	\frametitle{Consigli per raccontare la scienza}
	\begin{itemize}
		\item Ridurre al minimo, se non addirittura non utilizzare il gergo scientifico (possono venire in aiuto un glossario o le voci su Wikipedia)
		\onslide<2->{\item Bisogna raccontare una storia}
		\onslide<3->{\item Non bisogna lasciare spazio ad ambiguit� e cattive interpretazioni (possono venire in aiuto: abstract, uso del grassetto, suddivisione in sezioni)}
		\onslide<4->{\item Uso delle immagini: utilizzare, quando possibile, i grafici dei ricercatori o comunque immagini aderenti alla ricerca raccontata}
		\onslide<5->{\item Sintesi}
		\onslide<6->{\item Uso delle fonti: non solo l'articolo di partenza della nostra storia}
	\end{itemize}
\end{frame}
%
\end{document}
