\documentclass[12pt,a4paper]{article}
\usepackage{graphicx}

%\usepackage[latin1]{inputenc}
\usepackage[italian]{babel}
\selectlanguage{italian}
\usepackage{times}
\usepackage[utf8]{inputenc}
\usepackage[T1]{fontenc}
\usepackage{ifpdf}
\ifpdf
\usepackage[hyperindex]{hyperref}
\pdfadjustspacing=1
\fi
\title{Come scrivere un articolo divulgativo sulla scienza}
\author{Gianluigi Filippelli}
\date{08/05/2018}
%
%per i link
\usepackage{hyperref}
 \hypersetup{
  pdfpagemode=UseOutlines,
  %pdfstartview=FitV,
  bookmarksopen,
  bookmarksopenlevel=-1,
  pdftitle=,
  pdfauthor=,
  pdfsubject=,
  pdfkeywords=
  %pdfpagemode=FullScreen
  }
%
\textwidth 16cm
\textheight 22cm
\oddsidemargin -0.1cm
%
\title{Come scrivere un articolo divulgativo sulla scienza}
\author{Gianluigi Filippelli\footnote{Osservatorio Astronomico di Brera, Milano, Italy}}
\date{08/05/2018}
%
\begin{document}
 \maketitle
 
 Il blog è uno strumento potente e al tempo stesso divertente, perché da la possibilità di alternare post leggeri, come ad esempio quelli con i video incorporati, ad altri decisamente più lunghi, ricchi di spunti e di contenuti seri. I post appartenenti a quest'ultima tipologia appartengono, secondo me, alla schiera degli articoli seri, cui siamo abituati leggendo i quotidiani.\\
 Quando ci si propone sul web, però, si possono avere molti approcci, ma quando nello specifico si affronta l'argomento della scienza esiste, probabilmente, una strada maestra, che certo può essere percorsa in modi differenti, ma che dovrebbe essere la guida da tutti quelli che ambiscono a trattare di scienza.\\
 E' sempre, però, molto difficile identificare questa strada, ma una buona guida può essere l'articolo di \textbf{Dave Munger} ospitato sul blog di \textbf{Travis Saunders} e \textbf{Peter Janiszewski}\footnote{\href{http://scienceofblogging.com/how-to-write-a-good-research-blog-post/}{http://scienceofblogging.com/how-to-write-a-good-research-blog-post/}} e che farà da guida per queste mie brevi note. Ispirato a quanto scritto nell'articolo citato, vi proporrò alcuni consigli su come realizzare un buon post (o articolo) scientifico, centrato su una ricerca specifica (ma si potrebbe estendere anche ad argomenti più divulgativi o didattici).
 \begin{itemize}
 	\item \textbf{Scrivere di una data ricerca solo quando la si reputa interessante}, e questo non necessariamente ci lega al dover parlare di ricerca recente: è in un certo senso una definizione allargata di \emph{news}. Si può interpretare questo primo consiglio non solo come un modo per mantenere la libertà di scelta, ma anche come un invito a non fossilizzarsi su ciò che di nuovo propone il mondo della ricerca.
 	\item \textbf{Essere certi di aver compreso la ricerca di cui si vuole scrivere}. A mio modo di vedere non è necessario comprendere ogni dettaglio, perché spesso i dettagli sono comprensibili solo a chi lavora in quello stretto campo d'indagine, ma essere in grado di comprendere ciò che i ricercatori stanno raccontando. Una buona comprensione della ricerca rende molto più semplice e naturale lo svolgimento corretto dei consigli successivi.
 	\item \textbf{Mostrare perché la ricerca è interessante}, che implica non seguire necessariamente l'ordine seguito dai ricercatori nel loro \emph{paper}\footnote{Questo è l'articolo scientifico scritto per le riviste specialistiche, che chiamerò \emph{paper} per distinguerlo dall'\emph{articolo scientifico di stampo divulgativo}}, perché non è necessariamente quello migliore per raccontare la ricerca stessa.
 	\item \textbf{Lasciare che la ricerca parli per se stessa}, ovvero fornire dettagli su come la ricerca è stata condotta, o quanto meno dare un'idea sul metodo usato dai ricercatori.
 	\item Bisogna stare attenti a \textbf{non includere dettagli rilevanti solo per gli scienziati}, e quindi bisogna essere in grado di distinguere tra i dettagli utili al nostro racconto e quelli inutili, perché banalmente specialistici.
 	\item \textbf{Ridurre al minimo, se non addirittura non utilizzare il gergo scientifico}: a supporto dell'approfondimento del gergo si possono utilizzare i \emph{link} alle voci di Wikipedia o un glossario realizzato dallo stesso \emph{blogger} cui fare eventualmente riferimento.
 	\item Bisogna raccontare una storia, ma \textbf{non bisogna lasciare spazio ad ambiguità e cattive interpretazioni}, soprattutto nella fase introduttiva, visto che non necessariamente i lettori potrebbero leggere l'intero nostro articolo. Questa è certo una parte molto difficile da portare a termine, che può certo essere semplificata con l'uso di un'introduzione riassuntiva del nostro articolo (l'equivalente di un \emph{abstract} per un \emph{paper}). In alternativa o in aggiunta sono consigliati l'uso del grassetto o del corsivo, la suddivisione in sezioni, l'utilizzo delle immagini, non solo per illustrare l'articolo, ma anche per fornire delle pause di lettura.
 \end{itemize}
 Le immagini in particolare assumono un'importanza fondamentale: innanzitutto sarebbe bene utilizzare i grafici proposti dai ricercatori, se presenti, ricordandosi di spiegare ogni grafico presentato: un'operazione di questo genere ha anche il vantaggio di farci comprendere quanto della ricerca abbiamo noi stessi compreso. Non bisogna, poi, associare un'immagine generica con una ricerca specifica: non necessariamente questo è l'abbinamento corretto.\\
 Infine bisogna \textbf{cercare di essere il più concisi possibile} (se necessario suddividere il proprio articolo in due o più parti) e soprattutto \textbf{citare le fonti}, che non sono solo il \emph{paper} che abbiamo esaminato, ma anche quelli appartenenti alla bibliografia proposta dai ricercatori, a maggior ragione se per il nostro racconto abbiamo utilizzato anche altri articoli per approfondire l'argomento che stiamo trattando. Ovviamente la bibliografia del \emph{paper} è un ottimo punto di partenza insieme alla classica ricerca su Google.
\end{document}